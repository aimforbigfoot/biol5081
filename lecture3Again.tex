% Options for packages loaded elsewhere
\PassOptionsToPackage{unicode}{hyperref}
\PassOptionsToPackage{hyphens}{url}
%
\documentclass[
]{article}
\usepackage{amsmath,amssymb}
\usepackage{iftex}
\ifPDFTeX
  \usepackage[T1]{fontenc}
  \usepackage[utf8]{inputenc}
  \usepackage{textcomp} % provide euro and other symbols
\else % if luatex or xetex
  \usepackage{unicode-math} % this also loads fontspec
  \defaultfontfeatures{Scale=MatchLowercase}
  \defaultfontfeatures[\rmfamily]{Ligatures=TeX,Scale=1}
\fi
\usepackage{lmodern}
\ifPDFTeX\else
  % xetex/luatex font selection
\fi
% Use upquote if available, for straight quotes in verbatim environments
\IfFileExists{upquote.sty}{\usepackage{upquote}}{}
\IfFileExists{microtype.sty}{% use microtype if available
  \usepackage[]{microtype}
  \UseMicrotypeSet[protrusion]{basicmath} % disable protrusion for tt fonts
}{}
\makeatletter
\@ifundefined{KOMAClassName}{% if non-KOMA class
  \IfFileExists{parskip.sty}{%
    \usepackage{parskip}
  }{% else
    \setlength{\parindent}{0pt}
    \setlength{\parskip}{6pt plus 2pt minus 1pt}}
}{% if KOMA class
  \KOMAoptions{parskip=half}}
\makeatother
\usepackage{xcolor}
\usepackage[margin=1in]{geometry}
\usepackage{graphicx}
\makeatletter
\def\maxwidth{\ifdim\Gin@nat@width>\linewidth\linewidth\else\Gin@nat@width\fi}
\def\maxheight{\ifdim\Gin@nat@height>\textheight\textheight\else\Gin@nat@height\fi}
\makeatother
% Scale images if necessary, so that they will not overflow the page
% margins by default, and it is still possible to overwrite the defaults
% using explicit options in \includegraphics[width, height, ...]{}
\setkeys{Gin}{width=\maxwidth,height=\maxheight,keepaspectratio}
% Set default figure placement to htbp
\makeatletter
\def\fps@figure{htbp}
\makeatother
\setlength{\emergencystretch}{3em} % prevent overfull lines
\providecommand{\tightlist}{%
  \setlength{\itemsep}{0pt}\setlength{\parskip}{0pt}}
\setcounter{secnumdepth}{-\maxdimen} % remove section numbering
\ifLuaTeX
  \usepackage{selnolig}  % disable illegal ligatures
\fi
\usepackage{bookmark}
\IfFileExists{xurl.sty}{\usepackage{xurl}}{} % add URL line breaks if available
\urlstyle{same}
\hypersetup{
  pdftitle={Lecture 3},
  pdfauthor={Eryn McFarlane},
  hidelinks,
  pdfcreator={LaTeX via pandoc}}

\title{Lecture 3}
\author{Eryn McFarlane}
\date{}

\begin{document}
\maketitle

We're going to simulate some data that can be used in each of the
following tests.

The response variable is normally distributed, and then we're going to
back calculate some predictor variables.

NOTE: There are 9 questions below to answer. Do your best to answer them
in full sentences. This is part of what we're practicing in this class.

QUESTION: Describe the trait that you're simulating. This can be a real
trait that you are working on, or hope to work on, or it can be
completely made up. Pay special attention to the N that you're
expecting, the mean, and the sd around the mean.

\#``` \{r stimulate\} rnorm(1000, 0,
10)-\textgreater trait\_of\_interest \#\#\# change this to be a trait
that you're actually interested in, with an appropriate distribution!

plot(density(trait\_of\_interest)) min(trait\_of\_interest)
max(trait\_of\_interest) \#```

QUESTION: Describe the predictor variable. What does this mean
biologically to your trait of interest. How did you decide on the
numbers in yes on line 33?

yes\textless-sample(trait\_of\_interest, 500, replace=FALSE, prob =
ifelse(trait\_of\_interest\textgreater0, 0.95, 0.15)) \#\#\#\# play with
this line! Is the test statistically significant. When is it not?
predictor\_t.test\textless-(trait\_of\_interest \%in\% yes)

cbind.data.frame(trait\_of\_interest,
predictor\_t.test)-\textgreater data

mean(data{[}which(data\(predictor_t.test==TRUE),1])
mean(data[which(data\)predictor\_t.test==FALSE),1{]})
t.test(trait\_of\_interest\textasciitilde predictor\_t.test, data=data)
\#\#\# this does a two sample t-test. What would a one sample t test be
testing? How would you do that?

\subsubsection{plots our two samples for
distribution}\label{plots-our-two-samples-for-distribution}

plot(density(data{[}which(data\(predictor_t.test==FALSE),1]), col="red", main="Two sample t test")
lines(density(data[which(data\)predictor\_t.test==TRUE),1{]}), ylim=c(0,
0.1), xlim=c(-20,20), main=``Two Sample T test'')

\#\#\#plot one sample distribution
plot(density(data\(trait_of_interest), col="red", main="One sample t test")
t.test(data\)trait\_of\_interest) \#\#\# what is this test doing? `
QUESTION: Write one sentence where you report your t.text.

Next we're going to move to Anova. So, the first thing we'll do is break
our response variable (same one!) into 5 different categories, just as
we did for the t-tests.

QUESTION: Describe the predictor variable. What does this mean
biologically to your trait of interest. How did you decide on the
numbers in lines 60, 61, 62, 63?

test1\textless-sample(trait\_of\_interest, 200, replace=FALSE, prob =
ifelse(trait\_of\_interest\textgreater7, 0.95, 0.15))
test2\textless-sample(trait\_of\_interest{[}which(trait\_of\_interest
\%in\% test1 == FALSE){]}, 200, replace=FALSE, prob =
ifelse(trait\_of\_interest{[}which(trait\_of\_interest \%in\% test1 ==
FALSE){]} \textgreater4, 0.95, 0.15))
test3\textless-sample(trait\_of\_interest{[}which(trait\_of\_interest
\%in\% test2 == FALSE \textbar{} trait\_of\_interest \%in\% test1
==FALSE){]}, 200, replace=FALSE, prob =
ifelse(trait\_of\_interest{[}which(trait\_of\_interest \%in\% test2 ==
FALSE \textbar{} trait\_of\_interest \%in\% test1 == FALSE){]}
\textgreater0, 0.95, 0.15))
test4\textless-sample(trait\_of\_interest{[}which(trait\_of\_interest
\%in\% test2 == FALSE \textbar{} trait\_of\_interest \%in\% test1
==FALSE \textbar{} trait\_of\_interest \%in\% test3 == FALSE){]}, 200,
replace=FALSE, prob =
ifelse(trait\_of\_interest{[}which(trait\_of\_interest \%in\% test2 ==
FALSE \textbar{} trait\_of\_interest \%in\% test1 ==FALSE \textbar{}
trait\_of\_interest \%in\% test3 == FALSE){]} \textgreater-4, 0.95,
0.15)) test5\textless-trait\_of\_interest{[}which(trait\_of\_interest
\%in\% test1 == FALSE\textbar{} trait\_of\_interest \%in\% test2 ==
FALSE \textbar{} trait\_of\_interest \%in\% test3 == FALSE \textbar{}
trait\_of\_interest \%in\% test4 == FALSE){]}

plot(density(test1), ylim=c(0, 0.1), main=``Anovas'')
lines(density(test2), col=``red'') lines(density(test3), col=``blue'')
lines(density(test4), col=``purple'') lines(density(test5),
col=``yellow'')

anova\_predictor\textless-data.frame(ifelse(trait\_of\_interest \%in\%
test1 == TRUE, ``group1'', ifelse(trait\_of\_interest \%in\% test2 ==
TRUE, ``group2'', ifelse(trait\_of\_interest \%in\% test3 == TRUE,
``group3'', ifelse(trait\_of\_interest \%in\% test4 == TRUE, ``group4'',
``group5'')))))

data2\textless-cbind.data.frame(data, anova\_predictor)
names(data2)\textless-c(``trait\_of\_interest'', ``predictor\_t.test'',
``anova\_predictor'')

anova(aov(trait\_of\_interest\textasciitilde anova\_predictor,
data=data2)) \#\#\# what does this do?

anova(lm(trait\_of\_interest\textasciitilde anova\_predictor,
data=data2)) \#\#\# what does this do?
summary(aov(trait\_of\_interest\textasciitilde anova\_predictor,
data=data2)) \#\#\# what does this do? What do you notice about the last
three tests?

\subsubsection{what information is missing here that you wished you had
to understand your study
better?}\label{what-information-is-missing-here-that-you-wished-you-had-to-understand-your-study-better}

analysis\_of\_variance\textless-aov(trait\_of\_interest\textasciitilde anova\_predictor,
data=data2) \#\#name the model to keep it for downstream
TukeyHSD(analysis\_of\_variance, conf.level = 0.95) \#\#\# what does
this do, and where are the differences?

QUESTION: Write one sentence where you report your ANOVA and Tukey
tests. What did you find, and how do you report this?

Again, our simulations aren't to be actually causal, I'm simulating
predictor variables to fit a response I've already made. Normally, we
would simulate the response variables from more thoughtful predictor
variables.

QUESTION:what is the difference between the assumed distributions for
the prior predictor variables, and this one?

QUESTION: Describe the predictor variable. What does this mean
biologically to your trait of interest. How did you decide on the
numbers in line 104?

QUESTION: What is the difference between a regression and a correlation?
When would you use each? How does the test stat from the correlation
compare to the effect size from the regression? ```

QUESTION: Report your regression and correlation in a sentence.
Differentiate between them and what you report for each.

\end{document}
