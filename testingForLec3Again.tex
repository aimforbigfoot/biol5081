% Options for packages loaded elsewhere
\PassOptionsToPackage{unicode}{hyperref}
\PassOptionsToPackage{hyphens}{url}
%
\documentclass[
]{article}
\usepackage{amsmath,amssymb}
\usepackage{iftex}
\ifPDFTeX
  \usepackage[T1]{fontenc}
  \usepackage[utf8]{inputenc}
  \usepackage{textcomp} % provide euro and other symbols
\else % if luatex or xetex
  \usepackage{unicode-math} % this also loads fontspec
  \defaultfontfeatures{Scale=MatchLowercase}
  \defaultfontfeatures[\rmfamily]{Ligatures=TeX,Scale=1}
\fi
\usepackage{lmodern}
\ifPDFTeX\else
  % xetex/luatex font selection
\fi
% Use upquote if available, for straight quotes in verbatim environments
\IfFileExists{upquote.sty}{\usepackage{upquote}}{}
\IfFileExists{microtype.sty}{% use microtype if available
  \usepackage[]{microtype}
  \UseMicrotypeSet[protrusion]{basicmath} % disable protrusion for tt fonts
}{}
\makeatletter
\@ifundefined{KOMAClassName}{% if non-KOMA class
  \IfFileExists{parskip.sty}{%
    \usepackage{parskip}
  }{% else
    \setlength{\parindent}{0pt}
    \setlength{\parskip}{6pt plus 2pt minus 1pt}}
}{% if KOMA class
  \KOMAoptions{parskip=half}}
\makeatother
\usepackage{xcolor}
\usepackage[margin=1in]{geometry}
\usepackage{color}
\usepackage{fancyvrb}
\newcommand{\VerbBar}{|}
\newcommand{\VERB}{\Verb[commandchars=\\\{\}]}
\DefineVerbatimEnvironment{Highlighting}{Verbatim}{commandchars=\\\{\}}
% Add ',fontsize=\small' for more characters per line
\usepackage{framed}
\definecolor{shadecolor}{RGB}{248,248,248}
\newenvironment{Shaded}{\begin{snugshade}}{\end{snugshade}}
\newcommand{\AlertTok}[1]{\textcolor[rgb]{0.94,0.16,0.16}{#1}}
\newcommand{\AnnotationTok}[1]{\textcolor[rgb]{0.56,0.35,0.01}{\textbf{\textit{#1}}}}
\newcommand{\AttributeTok}[1]{\textcolor[rgb]{0.13,0.29,0.53}{#1}}
\newcommand{\BaseNTok}[1]{\textcolor[rgb]{0.00,0.00,0.81}{#1}}
\newcommand{\BuiltInTok}[1]{#1}
\newcommand{\CharTok}[1]{\textcolor[rgb]{0.31,0.60,0.02}{#1}}
\newcommand{\CommentTok}[1]{\textcolor[rgb]{0.56,0.35,0.01}{\textit{#1}}}
\newcommand{\CommentVarTok}[1]{\textcolor[rgb]{0.56,0.35,0.01}{\textbf{\textit{#1}}}}
\newcommand{\ConstantTok}[1]{\textcolor[rgb]{0.56,0.35,0.01}{#1}}
\newcommand{\ControlFlowTok}[1]{\textcolor[rgb]{0.13,0.29,0.53}{\textbf{#1}}}
\newcommand{\DataTypeTok}[1]{\textcolor[rgb]{0.13,0.29,0.53}{#1}}
\newcommand{\DecValTok}[1]{\textcolor[rgb]{0.00,0.00,0.81}{#1}}
\newcommand{\DocumentationTok}[1]{\textcolor[rgb]{0.56,0.35,0.01}{\textbf{\textit{#1}}}}
\newcommand{\ErrorTok}[1]{\textcolor[rgb]{0.64,0.00,0.00}{\textbf{#1}}}
\newcommand{\ExtensionTok}[1]{#1}
\newcommand{\FloatTok}[1]{\textcolor[rgb]{0.00,0.00,0.81}{#1}}
\newcommand{\FunctionTok}[1]{\textcolor[rgb]{0.13,0.29,0.53}{\textbf{#1}}}
\newcommand{\ImportTok}[1]{#1}
\newcommand{\InformationTok}[1]{\textcolor[rgb]{0.56,0.35,0.01}{\textbf{\textit{#1}}}}
\newcommand{\KeywordTok}[1]{\textcolor[rgb]{0.13,0.29,0.53}{\textbf{#1}}}
\newcommand{\NormalTok}[1]{#1}
\newcommand{\OperatorTok}[1]{\textcolor[rgb]{0.81,0.36,0.00}{\textbf{#1}}}
\newcommand{\OtherTok}[1]{\textcolor[rgb]{0.56,0.35,0.01}{#1}}
\newcommand{\PreprocessorTok}[1]{\textcolor[rgb]{0.56,0.35,0.01}{\textit{#1}}}
\newcommand{\RegionMarkerTok}[1]{#1}
\newcommand{\SpecialCharTok}[1]{\textcolor[rgb]{0.81,0.36,0.00}{\textbf{#1}}}
\newcommand{\SpecialStringTok}[1]{\textcolor[rgb]{0.31,0.60,0.02}{#1}}
\newcommand{\StringTok}[1]{\textcolor[rgb]{0.31,0.60,0.02}{#1}}
\newcommand{\VariableTok}[1]{\textcolor[rgb]{0.00,0.00,0.00}{#1}}
\newcommand{\VerbatimStringTok}[1]{\textcolor[rgb]{0.31,0.60,0.02}{#1}}
\newcommand{\WarningTok}[1]{\textcolor[rgb]{0.56,0.35,0.01}{\textbf{\textit{#1}}}}
\usepackage{graphicx}
\makeatletter
\def\maxwidth{\ifdim\Gin@nat@width>\linewidth\linewidth\else\Gin@nat@width\fi}
\def\maxheight{\ifdim\Gin@nat@height>\textheight\textheight\else\Gin@nat@height\fi}
\makeatother
% Scale images if necessary, so that they will not overflow the page
% margins by default, and it is still possible to overwrite the defaults
% using explicit options in \includegraphics[width, height, ...]{}
\setkeys{Gin}{width=\maxwidth,height=\maxheight,keepaspectratio}
% Set default figure placement to htbp
\makeatletter
\def\fps@figure{htbp}
\makeatother
\setlength{\emergencystretch}{3em} % prevent overfull lines
\providecommand{\tightlist}{%
  \setlength{\itemsep}{0pt}\setlength{\parskip}{0pt}}
\setcounter{secnumdepth}{-\maxdimen} % remove section numbering
\ifLuaTeX
  \usepackage{selnolig}  % disable illegal ligatures
\fi
\IfFileExists{bookmark.sty}{\usepackage{bookmark}}{\usepackage{hyperref}}
\IfFileExists{xurl.sty}{\usepackage{xurl}}{} % add URL line breaks if available
\urlstyle{same}
\hypersetup{
  pdftitle={Lecture 3 - Ahmed's Submission},
  pdfauthor={Ahmed Nadeem},
  hidelinks,
  pdfcreator={LaTeX via pandoc}}

\title{Lecture 3 - Ahmed's Submission}
\author{Ahmed Nadeem}
\date{2024-10-03}

\begin{document}
\maketitle

\hypertarget{im-not-sure-why-my-r-code-is-going-over-the-text-but-here-is-my-github-repo-for-this-course-the-file-for-this-is-called-testingforlec3again.rmd}{%
\subsection{I'm not sure why my R code is going over the text, but here
is my github repo for this course, the file for this is called
``testingForLec3Again.Rmd''}\label{im-not-sure-why-my-r-code-is-going-over-the-text-but-here-is-my-github-repo-for-this-course-the-file-for-this-is-called-testingforlec3again.rmd}}

We're going to simulate some data that can be used in each of the
following tests.

The response variable is normally distributed, and then we're going to
back calculate some predictor variables.

NOTE: There are 9 questions below to answer. Do your best to answer them
in full sentences. This is part of what we're practicing in this class.

\emph{QUESTION 1}: Describe the trait that you're simulating. This can
be a real trait that you are working on, or hope to work on, or it can
be completely made up. - Answer: I am working on the perception of
space, so that means I want to see how well people are able at
understanding what they look at and how well do they interact and move
in that place that they look at. So for the later part, I literally
measure how well you can place an item down at a given location (given
some handicap or sensation cue). And to see if there are biases in the
way you live/interact in the world, I have to measure how ``off'' you
were from going to a location. And that is what
\texttt{distanceFromTargetError} means in the little snippet below!

Pay special attention to the N that you're expecting, the mean, and the
sd around the mean.

\begin{Shaded}
\begin{Highlighting}[]
\FunctionTok{rnorm}\NormalTok{(}\DecValTok{500}\NormalTok{, }\FloatTok{2.5}\NormalTok{, }\DecValTok{5}\NormalTok{)}\OtherTok{{-}\textgreater{}}\NormalTok{distanceFromTargetError }\DocumentationTok{\#\#\# change this to be a trait that you\textquotesingle{}re actually interested in, with an appropriate distribution! }

\CommentTok{\# }

\FunctionTok{plot}\NormalTok{(}\FunctionTok{density}\NormalTok{(distanceFromTargetError))}
\end{Highlighting}
\end{Shaded}

\includegraphics{testingForLec3Again_files/figure-latex/simulate trait of interest-1.pdf}

\begin{Shaded}
\begin{Highlighting}[]
\FunctionTok{min}\NormalTok{(distanceFromTargetError)}
\end{Highlighting}
\end{Shaded}

\begin{verbatim}
## [1] -10.71907
\end{verbatim}

\begin{Shaded}
\begin{Highlighting}[]
\FunctionTok{max}\NormalTok{(distanceFromTargetError)}
\end{Highlighting}
\end{Shaded}

\begin{verbatim}
## [1] 17.91235
\end{verbatim}

\emph{QUESTION 2}: Describe the predictor variable. What does this mean
biologically to your trait of interest. How did you decide on the
numbers in yes on line 33? - Answer: Biologically speaking, this
\texttt{distanceFromTargetError} is me measuring the error from a given
location and the location chosen by a participant. In biological terms,
I am (hopefully) measuring the brain's internal representation of space
based on the cues it used to understand space AND THEN I will also be
measuring how well the motor response was at recreating and
adjusting/using that internal representation. ANd how did I decide it
lies on a line? well, if space itself lies on a line (3 lines cuz 3
axes) so that means it is natural for me to say that the error also lies
on a line (error = actual position - position placed by participant)

\begin{Shaded}
\begin{Highlighting}[]
\NormalTok{yes}\OtherTok{\textless{}{-}}\FunctionTok{sample}\NormalTok{(distanceFromTargetError, }\AttributeTok{size =} \DecValTok{30}\NormalTok{, }\AttributeTok{replace=}\ConstantTok{FALSE}\NormalTok{, }\AttributeTok{prob =} \FunctionTok{ifelse}\NormalTok{(}\AttributeTok{test =}\NormalTok{ distanceFromTargetError}\SpecialCharTok{\textgreater{}}\FloatTok{2.5}\NormalTok{, }\AttributeTok{yes =} \FloatTok{0.95}\NormalTok{,}\AttributeTok{no =}  \FloatTok{0.15}\NormalTok{)) }\DocumentationTok{\#\#\#\# play with this line! Is the test statistically significant. When is it not?}
\NormalTok{predictor\_t.test}\OtherTok{\textless{}{-}}\NormalTok{(distanceFromTargetError }\SpecialCharTok{\%in\%}\NormalTok{ yes)}

\FunctionTok{cbind.data.frame}\NormalTok{(distanceFromTargetError, predictor\_t.test)}\OtherTok{{-}\textgreater{}}\NormalTok{data}

\FunctionTok{mean}\NormalTok{(data[}\FunctionTok{which}\NormalTok{(data}\SpecialCharTok{$}\NormalTok{predictor\_t.test}\SpecialCharTok{==}\ConstantTok{TRUE}\NormalTok{),}\DecValTok{1}\NormalTok{])}
\end{Highlighting}
\end{Shaded}

\begin{verbatim}
## [1] 5.293438
\end{verbatim}

\begin{Shaded}
\begin{Highlighting}[]
\FunctionTok{mean}\NormalTok{(data[}\FunctionTok{which}\NormalTok{(data}\SpecialCharTok{$}\NormalTok{predictor\_t.test}\SpecialCharTok{==}\ConstantTok{FALSE}\NormalTok{),}\DecValTok{1}\NormalTok{])}
\end{Highlighting}
\end{Shaded}

\begin{verbatim}
## [1] 2.63836
\end{verbatim}

\begin{Shaded}
\begin{Highlighting}[]
\FunctionTok{t.test}\NormalTok{(distanceFromTargetError}\SpecialCharTok{\textasciitilde{}}\NormalTok{predictor\_t.test, }\AttributeTok{data=}\NormalTok{data) }\DocumentationTok{\#\#\# this does a two sample t{-}test. What would a one sample t test be testing? How would you do that?}
\end{Highlighting}
\end{Shaded}

\begin{verbatim}
## 
##  Welch Two Sample t-test
## 
## data:  distanceFromTargetError by predictor_t.test
## t = -3.7967, df = 36.296, p-value = 0.0005385
## alternative hypothesis: true difference in means between group FALSE and group TRUE is not equal to 0
## 95 percent confidence interval:
##  -4.072933 -1.237224
## sample estimates:
## mean in group FALSE  mean in group TRUE 
##            2.638360            5.293438
\end{verbatim}

\begin{Shaded}
\begin{Highlighting}[]
\DocumentationTok{\#\#\# plots our two samples for distribution}
\FunctionTok{plot}\NormalTok{(}\FunctionTok{density}\NormalTok{(data[}\FunctionTok{which}\NormalTok{(data}\SpecialCharTok{$}\NormalTok{predictor\_t.test}\SpecialCharTok{==}\ConstantTok{FALSE}\NormalTok{),}\DecValTok{1}\NormalTok{]), }\AttributeTok{col=}\StringTok{"red"}\NormalTok{, }\AttributeTok{main=}\StringTok{"Two sample t test"}\NormalTok{)}
\FunctionTok{lines}\NormalTok{(}\FunctionTok{density}\NormalTok{(data[}\FunctionTok{which}\NormalTok{(data}\SpecialCharTok{$}\NormalTok{predictor\_t.test}\SpecialCharTok{==}\ConstantTok{TRUE}\NormalTok{),}\DecValTok{1}\NormalTok{]), }\AttributeTok{ylim=}\FunctionTok{c}\NormalTok{(}\DecValTok{0}\NormalTok{, }\FloatTok{0.1}\NormalTok{), }\AttributeTok{xlim=}\FunctionTok{c}\NormalTok{(}\SpecialCharTok{{-}}\DecValTok{20}\NormalTok{,}\DecValTok{20}\NormalTok{), }\AttributeTok{main=}\StringTok{"Two Sample T test"}\NormalTok{)}
\end{Highlighting}
\end{Shaded}

\includegraphics{testingForLec3Again_files/figure-latex/simulate predictor variable for a t test-1.pdf}

\begin{Shaded}
\begin{Highlighting}[]
\DocumentationTok{\#\#\#plot one sample distribution}
\FunctionTok{plot}\NormalTok{(}\FunctionTok{density}\NormalTok{(data}\SpecialCharTok{$}\NormalTok{distanceFromTargetError), }\AttributeTok{col=}\StringTok{"red"}\NormalTok{, }\AttributeTok{main=}\StringTok{"One sample t test"}\NormalTok{)}
\end{Highlighting}
\end{Shaded}

\includegraphics{testingForLec3Again_files/figure-latex/simulate predictor variable for a t test-2.pdf}

\begin{Shaded}
\begin{Highlighting}[]
\FunctionTok{t.test}\NormalTok{(data}\SpecialCharTok{$}\NormalTok{distanceFromTargetError) }\DocumentationTok{\#\#\# what is this test doing?; t.test(data$distanceFromTargetError, mu = mean(distanceFromTargetError)) }\AlertTok{\#\#\#}\DocumentationTok{ what is this test doing?}
\end{Highlighting}
\end{Shaded}

\begin{verbatim}
## 
##  One Sample t-test
## 
## data:  data$distanceFromTargetError
## t = 12.724, df = 499, p-value < 2.2e-16
## alternative hypothesis: true mean is not equal to 0
## 95 percent confidence interval:
##  2.365668 3.229660
## sample estimates:
## mean of x 
##  2.797664
\end{verbatim}

\emph{QUESTION 3}: Write one sentence where you report your t.text. -
Answer: if I had a sample of 30 participants, where i picked these
samples based on that prob function(took me a while to understand LOL).
then i would not have statstically signifcant data cuz my SD from the
first code block is large. making the 2 samples VERY big, however, the
one sample t test, comparing to a mean of 0 (which is not my mean of my
data set.)

Next we're going to move to Anova. So, the first thing we'll do is break
our response variable (same one!) into 5 different categories, just as
we did for the t-tests.

\emph{QUESTION 4}: Describe the predictor variable. What does this mean
biologically to your trait of interest. How did you decide on the
numbers in lines 60, 61, 62, 63? - Answer: I picked the sample size
numbers based on how many participants I usually get in my studies
(around 30-50).

\begin{Shaded}
\begin{Highlighting}[]
\NormalTok{trait\_of\_interest }\OtherTok{=}\NormalTok{ distanceFromTargetError}
\NormalTok{test1}\OtherTok{\textless{}{-}}\FunctionTok{sample}\NormalTok{(trait\_of\_interest, }\DecValTok{20}\NormalTok{, }\AttributeTok{replace=}\ConstantTok{FALSE}\NormalTok{, }\AttributeTok{prob =} \FunctionTok{ifelse}\NormalTok{(trait\_of\_interest}\SpecialCharTok{\textgreater{}}\DecValTok{7}\NormalTok{, }\FloatTok{0.95}\NormalTok{, }\FloatTok{0.15}\NormalTok{))}
\NormalTok{test2}\OtherTok{\textless{}{-}}\FunctionTok{sample}\NormalTok{(trait\_of\_interest[}\FunctionTok{which}\NormalTok{(trait\_of\_interest }\SpecialCharTok{\%in\%}\NormalTok{ test1 }\SpecialCharTok{==} \ConstantTok{FALSE}\NormalTok{)], }\DecValTok{200}\NormalTok{, }\AttributeTok{replace=}\ConstantTok{FALSE}\NormalTok{, }\AttributeTok{prob =} \FunctionTok{ifelse}\NormalTok{(trait\_of\_interest[}\FunctionTok{which}\NormalTok{(trait\_of\_interest }\SpecialCharTok{\%in\%}\NormalTok{ test1 }\SpecialCharTok{==} \ConstantTok{FALSE}\NormalTok{)] }\SpecialCharTok{\textgreater{}}\DecValTok{4}\NormalTok{, }\FloatTok{0.95}\NormalTok{, }\FloatTok{0.15}\NormalTok{))}
\NormalTok{test3}\OtherTok{\textless{}{-}}\FunctionTok{sample}\NormalTok{(trait\_of\_interest[}\FunctionTok{which}\NormalTok{(trait\_of\_interest }\SpecialCharTok{\%in\%}\NormalTok{ test2 }\SpecialCharTok{==} \ConstantTok{FALSE} \SpecialCharTok{|}\NormalTok{ trait\_of\_interest }\SpecialCharTok{\%in\%}\NormalTok{ test1 }\SpecialCharTok{==}\ConstantTok{FALSE}\NormalTok{)], }\DecValTok{200}\NormalTok{, }\AttributeTok{replace=}\ConstantTok{FALSE}\NormalTok{, }\AttributeTok{prob =} \FunctionTok{ifelse}\NormalTok{(trait\_of\_interest[}\FunctionTok{which}\NormalTok{(trait\_of\_interest }\SpecialCharTok{\%in\%}\NormalTok{ test2 }\SpecialCharTok{==} \ConstantTok{FALSE} \SpecialCharTok{|}\NormalTok{ trait\_of\_interest }\SpecialCharTok{\%in\%}\NormalTok{ test1 }\SpecialCharTok{==} \ConstantTok{FALSE}\NormalTok{)] }\SpecialCharTok{\textgreater{}}\DecValTok{0}\NormalTok{, }\FloatTok{0.95}\NormalTok{, }\FloatTok{0.15}\NormalTok{))}
\NormalTok{test4}\OtherTok{\textless{}{-}}\FunctionTok{sample}\NormalTok{(trait\_of\_interest[}\FunctionTok{which}\NormalTok{(trait\_of\_interest }\SpecialCharTok{\%in\%}\NormalTok{ test2 }\SpecialCharTok{==} \ConstantTok{FALSE} \SpecialCharTok{|}\NormalTok{ trait\_of\_interest }\SpecialCharTok{\%in\%}\NormalTok{ test1 }\SpecialCharTok{==}\ConstantTok{FALSE} \SpecialCharTok{|}\NormalTok{ trait\_of\_interest }\SpecialCharTok{\%in\%}\NormalTok{ test3 }\SpecialCharTok{==} \ConstantTok{FALSE}\NormalTok{)], }\DecValTok{200}\NormalTok{, }\AttributeTok{replace=}\ConstantTok{FALSE}\NormalTok{, }\AttributeTok{prob =} \FunctionTok{ifelse}\NormalTok{(trait\_of\_interest[}\FunctionTok{which}\NormalTok{(trait\_of\_interest }\SpecialCharTok{\%in\%}\NormalTok{ test2 }\SpecialCharTok{==} \ConstantTok{FALSE} \SpecialCharTok{|}\NormalTok{ trait\_of\_interest }\SpecialCharTok{\%in\%}\NormalTok{ test1 }\SpecialCharTok{==}\ConstantTok{FALSE} \SpecialCharTok{|}\NormalTok{ trait\_of\_interest }\SpecialCharTok{\%in\%}\NormalTok{ test3 }\SpecialCharTok{==} \ConstantTok{FALSE}\NormalTok{)] }\SpecialCharTok{\textgreater{}{-}}\DecValTok{4}\NormalTok{, }\FloatTok{0.95}\NormalTok{, }\FloatTok{0.15}\NormalTok{))}
\NormalTok{test5}\OtherTok{\textless{}{-}}\NormalTok{trait\_of\_interest[}\FunctionTok{which}\NormalTok{(trait\_of\_interest }\SpecialCharTok{\%in\%}\NormalTok{ test1 }\SpecialCharTok{==} \ConstantTok{FALSE}\SpecialCharTok{|}\NormalTok{ trait\_of\_interest }\SpecialCharTok{\%in\%}\NormalTok{ test2 }\SpecialCharTok{==} \ConstantTok{FALSE} \SpecialCharTok{|}\NormalTok{ trait\_of\_interest }\SpecialCharTok{\%in\%}\NormalTok{ test3 }\SpecialCharTok{==} \ConstantTok{FALSE} \SpecialCharTok{|}\NormalTok{ trait\_of\_interest }\SpecialCharTok{\%in\%}\NormalTok{ test4 }\SpecialCharTok{==} \ConstantTok{FALSE}\NormalTok{)]}


\FunctionTok{plot}\NormalTok{(}\FunctionTok{density}\NormalTok{(test1), }\AttributeTok{ylim=}\FunctionTok{c}\NormalTok{(}\DecValTok{0}\NormalTok{, }\FloatTok{0.1}\NormalTok{), }\AttributeTok{main=}\StringTok{"Anovas"}\NormalTok{)}
\FunctionTok{lines}\NormalTok{(}\FunctionTok{density}\NormalTok{(test2), }\AttributeTok{col=}\StringTok{"red"}\NormalTok{)}
\FunctionTok{lines}\NormalTok{(}\FunctionTok{density}\NormalTok{(test3), }\AttributeTok{col=}\StringTok{"blue"}\NormalTok{)}
\FunctionTok{lines}\NormalTok{(}\FunctionTok{density}\NormalTok{(test4), }\AttributeTok{col=}\StringTok{"purple"}\NormalTok{)}
\FunctionTok{lines}\NormalTok{(}\FunctionTok{density}\NormalTok{(test5), }\AttributeTok{col=}\StringTok{"yellow"}\NormalTok{)}
\end{Highlighting}
\end{Shaded}

\includegraphics{testingForLec3Again_files/figure-latex/Anova-1.pdf}

\begin{Shaded}
\begin{Highlighting}[]
\NormalTok{anova\_predictor}\OtherTok{\textless{}{-}}\FunctionTok{data.frame}\NormalTok{(}\FunctionTok{ifelse}\NormalTok{(trait\_of\_interest }\SpecialCharTok{\%in\%}\NormalTok{ test1 }\SpecialCharTok{==} \ConstantTok{TRUE}\NormalTok{, }\StringTok{"group1"}\NormalTok{, }\FunctionTok{ifelse}\NormalTok{(trait\_of\_interest }\SpecialCharTok{\%in\%}\NormalTok{ test2 }\SpecialCharTok{==} \ConstantTok{TRUE}\NormalTok{, }\StringTok{"group2"}\NormalTok{, }\FunctionTok{ifelse}\NormalTok{(trait\_of\_interest }\SpecialCharTok{\%in\%}\NormalTok{ test3 }\SpecialCharTok{==} \ConstantTok{TRUE}\NormalTok{, }\StringTok{"group3"}\NormalTok{, }\FunctionTok{ifelse}\NormalTok{(trait\_of\_interest }\SpecialCharTok{\%in\%}\NormalTok{ test4 }\SpecialCharTok{==} \ConstantTok{TRUE}\NormalTok{, }\StringTok{"group4"}\NormalTok{, }\StringTok{"group5"}\NormalTok{)))))}

\NormalTok{data2}\OtherTok{\textless{}{-}}\FunctionTok{cbind.data.frame}\NormalTok{(data, anova\_predictor)}
\FunctionTok{names}\NormalTok{(data2)}\OtherTok{\textless{}{-}}\FunctionTok{c}\NormalTok{(}\StringTok{"trait\_of\_interest"}\NormalTok{, }\StringTok{"predictor\_t.test"}\NormalTok{, }\StringTok{"anova\_predictor"}\NormalTok{)}


\FunctionTok{anova}\NormalTok{(}\FunctionTok{aov}\NormalTok{(trait\_of\_interest}\SpecialCharTok{\textasciitilde{}}\NormalTok{anova\_predictor, }\AttributeTok{data=}\NormalTok{data2)) }\DocumentationTok{\#\#\# what does this do?}
\end{Highlighting}
\end{Shaded}

\begin{verbatim}
## Analysis of Variance Table
## 
## Response: trait_of_interest
##                  Df Sum Sq Mean Sq F value    Pr(>F)    
## anova_predictor   4 2403.9  600.97    30.8 < 2.2e-16 ***
## Residuals       495 9658.3   19.51                      
## ---
## Signif. codes:  0 '***' 0.001 '**' 0.01 '*' 0.05 '.' 0.1 ' ' 1
\end{verbatim}

\begin{Shaded}
\begin{Highlighting}[]
\FunctionTok{anova}\NormalTok{(}\FunctionTok{lm}\NormalTok{(trait\_of\_interest}\SpecialCharTok{\textasciitilde{}}\NormalTok{anova\_predictor, }\AttributeTok{data=}\NormalTok{data2)) }\DocumentationTok{\#\#\# what does this do? }
\end{Highlighting}
\end{Shaded}

\begin{verbatim}
## Analysis of Variance Table
## 
## Response: trait_of_interest
##                  Df Sum Sq Mean Sq F value    Pr(>F)    
## anova_predictor   4 2403.9  600.97    30.8 < 2.2e-16 ***
## Residuals       495 9658.3   19.51                      
## ---
## Signif. codes:  0 '***' 0.001 '**' 0.01 '*' 0.05 '.' 0.1 ' ' 1
\end{verbatim}

\begin{Shaded}
\begin{Highlighting}[]
\FunctionTok{summary}\NormalTok{(}\FunctionTok{aov}\NormalTok{(trait\_of\_interest}\SpecialCharTok{\textasciitilde{}}\NormalTok{anova\_predictor, }\AttributeTok{data=}\NormalTok{data2)) }\DocumentationTok{\#\#\# what does this do? What do you notice about the last three tests?}
\end{Highlighting}
\end{Shaded}

\begin{verbatim}
##                  Df Sum Sq Mean Sq F value Pr(>F)    
## anova_predictor   4   2404   601.0    30.8 <2e-16 ***
## Residuals       495   9658    19.5                   
## ---
## Signif. codes:  0 '***' 0.001 '**' 0.01 '*' 0.05 '.' 0.1 ' ' 1
\end{verbatim}

\begin{Shaded}
\begin{Highlighting}[]
\DocumentationTok{\#\#\# what information is missing here that you wished you had to understand your study better?}
\NormalTok{analysis\_of\_variance}\OtherTok{\textless{}{-}}\FunctionTok{aov}\NormalTok{(trait\_of\_interest}\SpecialCharTok{\textasciitilde{}}\NormalTok{anova\_predictor, }\AttributeTok{data=}\NormalTok{data2) }\DocumentationTok{\#\#name the model to keep it for downstream}
\FunctionTok{TukeyHSD}\NormalTok{(analysis\_of\_variance, }\AttributeTok{conf.level =} \FloatTok{0.95}\NormalTok{) }\DocumentationTok{\#\#\# what does this do, and where are the differences?}
\end{Highlighting}
\end{Shaded}

\begin{verbatim}
##   Tukey multiple comparisons of means
##     95% family-wise confidence level
## 
## Fit: aov(formula = trait_of_interest ~ anova_predictor, data = data2)
## 
## $anova_predictor
##                    diff        lwr         upr     p adj
## group2-group1 -3.625755  -6.461972 -0.78953769 0.0045977
## group3-group1 -5.130928  -8.072807 -2.18904943 0.0000233
## group4-group1 -6.891550  -9.973073 -3.81002732 0.0000000
## group5-group1 -8.565449 -11.518271 -5.61262624 0.0000000
## group3-group2 -1.505173  -2.944996 -0.06535003 0.0353797
## group4-group2 -3.265795  -4.972905 -1.55868559 0.0000024
## group5-group2 -4.939694  -6.401747 -3.47764014 0.0000000
## group4-group3 -1.760622  -3.638052  0.11680841 0.0780517
## group5-group3 -3.434521  -5.092267 -1.77677434 0.0000002
## group5-group4 -1.673899  -3.568431  0.22063428 0.1119188
\end{verbatim}

\emph{QUESTION 5}: Write one sentence where you report your ANOVA and
Tukey tests. What did you find, and how do you report this? - Answer:
The ANOVA test shows me that there is a lot of variablity that is not
explained by the groups (lots of noise). This is because the resdiuals
value is large. Then there is als oa large F value, which shows me that
there is a large diff b/w the groups relative to their variablity.

Again, our simulations aren't to be actually causal, I'm simulating
predictor variables to fit a response I've already made. Normally, we
would simulate the response variables from more thoughtful predictor
variables.

\emph{QUESTION 6}:what is the difference between the assumed
distributions for the prior predictor variables, and this one? - Answer:
the difference is that for the 5 categories made for the anova are based
on the data set itself, but the last 2 were just based on random chance
(if i understand the code correctly).

\emph{QUESTION 7}: Describe the predictor variable. What does this mean
biologically to your trait of interest. How did you decide on the
numbers in line 104? - Answer: IT would show me an aspect that is
realted to the mottor inaccuracy in humans, such as distance or
percentage of a cue shown to an individual/participant.

\emph{QUESTION 8}: What is the difference between a regression and a
correlation? When would you use each? How does the test stat from the
correlation compare to the effect size from the regression? - Answer: 1.
Regresssion is used to tell how associated two variables are, and it
tells me how to predict them with a line of best fit 2. Correlation are
used to tell me what direction and how tightly the two variables
``travel'' together so to speak (strong or weak) and if it going up, or
down on a graph (positive and negative correlations ) 3. I would use a
correlation when I want to know if there is a trend in any data set, i
would use a regression to predict one variable against another. 4. The
correlation (0.977) and r\^{}2 of 0.9954 show me that there is a strong
correlation and predictive power in my data set.

\begin{Shaded}
\begin{Highlighting}[]
\NormalTok{linear\_regression\_predictor}\OtherTok{\textless{}{-}}\FloatTok{12.5}\SpecialCharTok{*}\NormalTok{trait\_of\_interest}\SpecialCharTok{+}\FunctionTok{rnorm}\NormalTok{(}\DecValTok{1000}\NormalTok{, }\DecValTok{0}\NormalTok{, }\DecValTok{4}\NormalTok{) }\DocumentationTok{\#\#\# change these numbers!! Remember that this is backwards from how we did this on day 1, so the slope should go the other way!}
\NormalTok{data3}\OtherTok{\textless{}{-}}\FunctionTok{cbind}\NormalTok{(data2, linear\_regression\_predictor)}

\NormalTok{lm}\OtherTok{\textless{}{-}}\FunctionTok{lm}\NormalTok{(trait\_of\_interest}\SpecialCharTok{\textasciitilde{}}\NormalTok{linear\_regression\_predictor, }\AttributeTok{data=}\NormalTok{data3)}
\FunctionTok{summary}\NormalTok{(lm)}\DocumentationTok{\#\#\# what is the output here? What are we interested in understanding from this model? How do we get m? How do we get the intercept?}
\end{Highlighting}
\end{Shaded}

\begin{verbatim}
## 
## Call:
## lm(formula = trait_of_interest ~ linear_regression_predictor, 
##     data = data3)
## 
## Residuals:
##      Min       1Q   Median       3Q      Max 
## -1.01038 -0.20705 -0.00808  0.20327  1.04970 
## 
## Coefficients:
##                              Estimate Std. Error t value Pr(>|t|)    
## (Intercept)                 0.0156139  0.0113936    1.37    0.171    
## linear_regression_predictor 0.0798536  0.0001614  494.68   <2e-16 ***
## ---
## Signif. codes:  0 '***' 0.001 '**' 0.01 '*' 0.05 '.' 0.1 ' ' 1
## 
## Residual standard error: 0.3133 on 998 degrees of freedom
## Multiple R-squared:  0.9959, Adjusted R-squared:  0.9959 
## F-statistic: 2.447e+05 on 1 and 998 DF,  p-value: < 2.2e-16
\end{verbatim}

\begin{Shaded}
\begin{Highlighting}[]
\NormalTok{eq }\OtherTok{=} \FunctionTok{paste0}\NormalTok{(}\StringTok{"y = "}\NormalTok{, }\FunctionTok{round}\NormalTok{(lm}\SpecialCharTok{$}\NormalTok{coefficients[}\DecValTok{2}\NormalTok{],}\DecValTok{1}\NormalTok{), }\StringTok{"*x"}\NormalTok{, }\StringTok{"+"}\NormalTok{,}\FunctionTok{round}\NormalTok{(lm}\SpecialCharTok{$}\NormalTok{coefficients[}\DecValTok{1}\NormalTok{],}\DecValTok{1}\NormalTok{), }\StringTok{", R\^{}2="}\NormalTok{, }\FunctionTok{round}\NormalTok{(}\FunctionTok{summary}\NormalTok{(lm)}\SpecialCharTok{$}\NormalTok{adj.r.squared, }\AttributeTok{digits=}\DecValTok{3}\NormalTok{))}
\FunctionTok{plot}\NormalTok{(data3}\SpecialCharTok{$}\NormalTok{linear\_regression\_predictor, data3}\SpecialCharTok{$}\NormalTok{trait\_of\_interest, }\AttributeTok{col=}\StringTok{"red"}\NormalTok{, }\AttributeTok{main=}\NormalTok{eq)}
\FunctionTok{abline}\NormalTok{(lm, }\AttributeTok{col=}\StringTok{"black"}\NormalTok{)}
\end{Highlighting}
\end{Shaded}

\includegraphics{testingForLec3Again_files/figure-latex/Linear Regression and correlation-1.pdf}

\begin{Shaded}
\begin{Highlighting}[]
\DocumentationTok{\#\#\# with the same data:}
\FunctionTok{cor.test}\NormalTok{(data3}\SpecialCharTok{$}\NormalTok{trait\_of\_interest, data3}\SpecialCharTok{$}\NormalTok{linear\_regression\_predictor) }\DocumentationTok{\#\#\# compare the sample estimate to the 1) simulated effect sizes and 2) to the estimated effect size }
\end{Highlighting}
\end{Shaded}

\begin{verbatim}
## 
##  Pearson's product-moment correlation
## 
## data:  data3$trait_of_interest and data3$linear_regression_predictor
## t = 494.68, df = 998, p-value < 2.2e-16
## alternative hypothesis: true correlation is not equal to 0
## 95 percent confidence interval:
##  0.9976986 0.9982042
## sample estimates:
##      cor 
## 0.997967
\end{verbatim}

\begin{Shaded}
\begin{Highlighting}[]
\DocumentationTok{\#\#\# how does the correlation estimate change when you change line 104?}
\end{Highlighting}
\end{Shaded}

\emph{QUESTION 9}: Report your regression and correlation in a sentence.
Differentiate between them and what you report for each. - Answer: The
linear regression model indicated that for every unit increase in
linear\_regression\_predictor, the trait\_of\_interest increases by
0.079 units (R² = 0.9954, p \textless{} 2e-16). The correlation between
the two variables was 0.9977 (p \textless{} 2e-16), indicating a strong
positive relationship.

\end{document}
